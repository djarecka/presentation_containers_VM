\documentclass[t,svgnames]{beamer}
\usepackage[utf8]{inputenc}
\usepackage{eurosym}
\usepackage{tikz,latexsym}
\usepackage{bm}
\usepackage{tikz-qtree}
\tikzset{
    invisible/.style={opacity=0},
    visible on/.style={alt={#1{}{invisible}}},
    alt/.code args={<#1>#2#3}{%
      \alt<#1>{\pgfkeysalso{#2}}{\pgfkeysalso{#3}} % \pgfkeysalso doesn't change the path
    },
}

\definecolor{links}{HTML}{2A1B81}
\hypersetup{colorlinks,linkcolor=,urlcolor=links}


\usepackage{caption}
\captionsetup[figure]{labelformat=empty}% redefines the caption setup of the figures environment

\newcommand{\prog}[1]{{\rm\bf#1}}

\newcommand\ftitle[1]{
  \frametitle{{\color{FireBrick}#1}\\
    \vspace{-.05\textheight}
    \makebox[.9666\linewidth]{\rule{\paperwidth}{0.4pt}}
  }
}

\usecolortheme{dove}
\usenavigationsymbolstemplate{}

%\usepackage[T1]{fontenc}
%\usepackage{lmodern}

\usepackage{relsize}
\usepackage{fancyvrb}
\fvset{frame=single,framerule=.2mm,fontsize=\relscale{.86}}
%\input{pygments}

\usepackage{float}
\newfloat{Listing}{hbtp}{lop}[section]
% dj: to mi cos przeszkadzalo z minted
%\newcommand*\FancyVerbStartString{}
%\newcommand*\FancyVerbStopString{}
%\newcommand{\code}[3]{% args: file, listing no., gobble value
%  \fvset{gobble=#3}%
%  \renewcommand*\FancyVerbStartString{\PY{c+c1}{//\PYZlt{}listing\PYZhy{}#2\PYZgt{}}}%
%  \renewcommand*\FancyVerbStopString{\PY{c+c1}{//\PYZlt{}/listing\PYZhy{}#2\PYZgt{}}}%
%  \input{#1}%
%}

\setbeamertemplate{blocks}[rounded][shadow=true]
\setbeamercolor{block title}{fg=White,bg=Black}
\setbeamercolor{block body}{bg=Beige}

\usepackage[overlay,absolute]{textpos}
\setlength{\TPHorizModule}{1cm}
\setlength{\TPVertModule}{1cm}


\title{Containers: Vagrant, Docker, Singularity}

\author{Dorota Jarecka: djarecka@mit.edu}
%\institute{Institute of Geophysics, Faculty of Physics, University of Warsaw}

\date{\small Brainhack Global 2017 -- Boston}

\begin{document}
  \frame{\maketitle}

  \frame<1->{
    \ftitle{Why do we need containers?}
    \vspace{1em}
      \begin{itemize}
        \item<1->{Science Reproducibility}
          \begin{itemize}
          \item<2->{Each project in a lab depends on complex software environments}
            \begin{itemize}
            \item<2->{operating system}
            \item<2->{drivers}
            \item<2->{software dependencies: Python/MATLAB/R + libraries}
          \end{itemize}
            \vspace{2em}
        \item<3>{We try to avoid}
            \begin{itemize}
            \item<3->{\it the computer I used was shut down a year ago, can't rerun the results from my publication...}
            \item<3->{\it the analysis were run by my student, have no idea where and how...}
            \item<3->{etc.}
          \end{itemize}
          \end{itemize}
      \end{itemize}
  }   

  \frame<1->{
    \ftitle{Why do we need containers?}
    \vspace{1em}
      \begin{itemize}
        \item<1->{Collaboration wit your colleagues}
          \begin{itemize}
          \item<2->{Sharing your code or using a repository might not be enough}

            \vspace{2em}

        \item<3>{We try to avoid}
            \begin{itemize}
            \item<3->{\it well, I forgot to mention that you have to use Clang, gcc never worked for me...}
            \item<3->{\it don't see any reason why it shouldn't work on Windows... (I actually have no idea about Windows, but won't say it...)}
            \item<3->{\textit{\textbf{it works on my computer...}}}
            \item<3->{etc.}
          \end{itemize}
          \end{itemize}
      \end{itemize}
  }


  \frame<1->{
    \ftitle{Why do we need containers?}
      \begin{itemize}
        \item<1->{Freedom to experiment!}
          \uncover<2->{
            \begin{figure}
          \centering
          \pgfimage[width=.4\textwidth]{img/universal_install_script_2x.png}
          \caption{\small{Universal Install Script from xkcd: {\it The failures {\bf usually} don't hurt anything...} {\bf Usually} all your old programs work...}}
        \end{figure}
            }
          \begin{itemize}

        \item<3>{We try to avoid}
            \begin{itemize}
            \item<3->{\it I just want to Undo the last five hours of my life...}
          \end{itemize}
          \end{itemize}
      \end{itemize}

  }






   \frame{
      \ftitle{Virtual Machines and Container Technologies}
      \begin{itemize}
        \vspace{1em}
        \item<1->{Main idea: isolate the computing environment
          \begin{itemize}
          \item<1->{Allow regenerating computing environments}
          \item<1->{Allow sharing your computing environments}
          \end{itemize}
        }
        \vspace{1em}
        \item<2->{Two types:
          \begin{itemize}
          \item<2->{Virtual Machines
            \begin{itemize}
            \item<3->{\href{https://www.virtualbox.org/}{Virtualbox}}
            \item<3->{\href{http://www.vmware.com/}{VMware}}
            \item<3->{\href{http://docs.aws.amazon.com/AWSEC2/latest/UserGuide/AMIs.html}{AWS}, \href{https://cloud.google.com/compute/}{Google Compute}, ...}
            \end{itemize}
          }
          \item<2->{Containers
            \begin{itemize}
            \item<4->{\href{https://www.docker.com/}{Docker}}
            \item<4->{\href{https://runc.io/}{runc}}
            \item<4->{\href{https://linuxcontainers.org/}{lxc/lxd}}
            \item<4->{\href{http://singularity.lbl.gov/}{Singularity}}
            \end{itemize}

          }
          \end{itemize}
        }

        \vspace{1em}
        \item<5->{The details differ (and matter depending on application)}
      \end{itemize}
   }

   \frame{
      \ftitle{Virtual Machines vs Container}
      \uncover<1->{
        \begin{figure}
          \centering
          \pgfimage[width=.8\textwidth]{img/Virtual_Machine_vs_Docker_Container.png}
          %\caption{\tiny{http://www.bogotobogo.com/DevOps/Docker/images/Docker_vs_Virtual_Machine/Virtual_Machine_vs_Docker_Container.png}}
        \end{figure}
      }
   }


   \frame{
     \ftitle{Vagrant}
     \vspace{1em}
     \begin{itemize}
        \item<1->{is a command line utility for managing virtual machines}
        \item<1->{provides easy to configure, reproducible, and portable work environments}
        \item<1->{on top of VirtualBox, VMware, AWS, or any other provider (VirtualBox is the default machine)}
     \end{itemize}
   }


   \frame{
     \ftitle{Vagrant: Creating a computing environments using}
     \vspace{1em}
     \begin{itemize}
        \item<1->{Using an existing box -- ubuntu/trusty64
           \uncover<2->{
           \vspace{2em}
           \begin{block}{}
           \$ mkdir vm \\
           \$ cd vm \\
           \$ vagrant init ubuntu/trusty64 \\
           \$ vagrant up ~~~\# the same as vagrant up --provider virtualbox \\
           \$ vagrant ssh  ~~~\# or vagrant ssh -c /bin/sh
           \end{block}}
        }
     \end{itemize}
   }

   \frame{
     \ftitle{Vagrant: installing new software}
     \vspace{1em}
     \begin{itemize}
        \item<1->{installing Emacs\\
          \vspace{0.5em} 
           \uncover<2->{within the VM:
           \begin{block}{}
           \$ sudo apt-get update \\
           \$ sudo apt-get install emacs \\
           \$ emacs \\ 
           \$ exit
           \end{block}}
        }
          \vspace{2em}
        \item<2->{installing FSL\\
          \vspace{0.5em}
          \begin{itemize}
          \item<2->{follow the instruction from {\href{http://neuro.debian.net/install_pkg.html?p=fsl-complete}{NeuroDebian}} (once it's fixed)}
          \end{itemize}
          }
     \end{itemize}
   }

   \frame{
     \ftitle{Docker}
     \vspace{1em}
     \begin{itemize}
        \item{leading software container platform}
        \item{an open-source project}
        \item{bundle libraries and required settings only (not a full operating system)}
        \item{more lightweight to deploy and faster to start up than VM}
        \vspace{4em}
        \item<2->{testing your Docker installation:
          \vspace{1em}
          \begin{block}{}
            \$ docker run hello-world \\
          \end{block}}

      \end{itemize}
   }



   \frame{
     \ftitle{Docker}
     \vspace{1em}
     \begin{itemize}
        \item<1->{Using existing images
          \vspace{1em}
          \begin{block}{}
            \$ docker pull busybox \\
            \$ docker images \\
            % todo rozbic to moze jakos??
            \$ docker run busybox \\
            \$ docker run busybox echo ``hello from busybox'' \\
            \$ docker run -it busybox sh \\
            \$ docker ps \\
            \$ docker rm \\
            %or
            \$ docker run -it \texttt{--}rm busybox \\
            \$ docker run -it \texttt{--}rm -v YourDirectory:/src busybox \\
          \end{block}}


     \end{itemize}
   }

   \frame{
     \ftitle{Docker: Installing software with Dockerfile}
     \vspace{1em}
     \begin{itemize}
        \item<1->{Dockerfile content
          \begin{block}{}
          FROM bids/base\_fsl \\
          RUN apt-get update -y \&\& apt-get install -y r-base
          \end{block}}
        \item<2->{Building a new container
          \vspace{1em}
          \begin{block}{}
            \$ docker build -t fslR .
          \end{block}}
        \item<3->{Running your new container
          \vspace{1em}
          \begin{block}{}
            \$ docker run -ti \texttt{--}rm fslR
          \end{block}}

      \end{itemize}
   }


   \frame{
     \ftitle{Docker and Nipype}
     \vspace{1em}
     \begin{itemize}
       \item{Using Nipype official Docker image}
       \vspace{1em}
          \uncover<2->{
          \begin{block}{}
            \$ docker images \\
            \$ docker pull nipype/nipype \\ %\#only if you haven't done it before (it's 5G) \\
            \$ docker images \\
            \$ docker run -it \texttt{--}rm nipype/nipype \\
          \end{block}}
          \vspace{1em}
          \uncover<3->{
          \begin{itemize}
           \item{within the nipype container}
          \end{itemize}
          \begin{block}{}
            \$ cd /src/nipype/nipype/pipeline/engine/ \\
            \$ py.test
          \end{block}}


     \end{itemize}
   }


   \frame{
     \ftitle{Docker: Docker Hub}
     \vspace{1em}
     Docker Hub is a cloud-based registry service which allows you to link to code repositories, build your images and share them. 
     \vspace{3em}
     \begin{itemize}
        \item<1->{\href{https://hub.docker.com/r/nipype/nipype/}{Nipype}}
        \item<2->{\href{https://hub.docker.com/r/continuumio/anaconda/}{Anaconda - a python distribution}}
        \item<3->{\href{https://hub.docker.com/r/continuumio/miniconda/}{Miniconda - a package manager}}
     \end{itemize}
   }

   \frame{
     \ftitle{Singularity}
     \vspace{1em}
     \begin{itemize}
        \item<1->{a container solution created for scientific and application driven workloads.}
        \item<2->{supports existing and traditional HPC resources}
        \item<3->{a user inside a Singularity container is the same user as outside the container}
        \item<4->{can use Vagrant to create containers (you have root privileges on your VM!)}
        \item<5->{can run existing Docker containers}
        \item<6->{Satra's presentation -- {\href{http://satra.cogitatum.org/om-images/}{Singularity on Openmind}}}
     \end{itemize}
   }




\end{document}
